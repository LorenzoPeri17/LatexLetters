\documentclass{responseletter}

%%%%%%%%%%%%%%%%%%%%%%%%%%%%%%%%%%%%%%%%%%%%%%%%%
%%%%%%%%%%%%%%%%%  Example setup %%%%%%%%%%%%%%%%
%%%%%%%%%%%%%%%%%%%%%%%%%%%%%%%%%%%%%%%%%%%%%%%%%
\begin{filecontents}[overwrite]{bibliography.bib}
    @book{Knuth_1986, address={Reading, Mass}, series={Computers and typesetting}, title={The TeXbook}, ISBN={978-0-201-13447-6}, callNumber={Z253.4.T47 K58 1986}, publisher={Addison-Wesley}, author={Knuth, Donald Ervin}, year={1986}, collection={Computers and typesetting} }
\end{filecontents}

\begin{filecontents}[overwrite]{original-document.tex}
\documentclass{article}
\usepackage{nameref}
\begin{document}
\section{Original Section}\label{section}
\end{document}
\end{filecontents}

\begin{filecontents}[overwrite]{revised-document.tex}
\documentclass{article}
\usepackage{nameref}
\begin{document}
\section{Revised Section}\label{section}
\end{document}
\end{filecontents}
%%%%%%%%%%%%%%%%%%%%%%%%%%%%%%%%%%%%%%%%%%%%%%%%%
%%%%%%%%% If something does not work %%%%%%%%%%%%
%%%%%%%%%%% don't forget to compile %%%%%%%%%%%%%
%%%%%%%%%%%%% original-document.tex %%%%%%%%%%%%%
%%%%%%%%%%%%%%%%%%%%%% and %%%%%%%%%%%%%%%%%%%%%%
%%%%%%%%%%%%% revised-document.tex %%%%%%%%%%%%%%
%%%%%%%%%%%%%%%%%%%%%%%%%%%%%%%%%%%%%%%%%%%%%%%%%

\address{
% TC:ignore
From Street\\
From City\\
From Postcode\\
From Country
% TC:endignore
}

\signature{Lorenzo Peri}

\headerlogo{\includegraphics[height=2cm]{LetterFiles/logo.png}}

%% include references from the manuscript
\external[original-]{original-document}
\external[revised-]{revised-document}


\begin{document}

\begin{letter}{
% TC:ignore
Journal Street\\
Journal City\\
Journal Postcode\\
Journal Country
% TC:endignore
}

% TC:ignore
\opening{Dear Editor,}

Thank you for forwarding the reports on our manuscript entitled \textbf{manuscript title}. 

We include a full response to the referees further below but here we highlight the major changes:

\begin{itemize}
    \item Point 1\\
    \item Point 2
\end{itemize}

\closing{Yours sincerely,}

\newref{Referee 1}

\referee{General points.}

General Response.

\begin{enumerate}
    \item 
    \referee{First point.}

    First Response.

    \item 
    \referee{Second point.}

    Second Response.
\end{enumerate}

\newref{Referee 2}

(The mean one)

\begin{enumerate}
    \item 
    \referee{Can you have figures?}

    We can have figures in the text.

    \begin{figure}
        \includegraphics[height=2cm]{LetterFiles/logo.png}
        \caption{We can have figures!}
    \end{figure}

    \item 
    \referee{Can you have tables?}

    And we can have also tables.
    \begin{table}
        \begin{tabular}{ |c|c|c| } 
            \hline
            cell1 & cell2 & cell3 \\ 
            \hline
            cell4 & cell5 & cell6 \\ 
            \hline
        \end{tabular}
        \caption{We can have tables!}
    \end{table}

    \item 
    \referee{Can you have citations?}

    Yes, we can even have citations. Here is the \TeX book  \cite{Knuth_1986}!

    \item 
    \referee{Can you reference sections or equations form the manuscript?}

    Yes I can use \texttt{\textbackslash external[prefix]\{document-name\}} to reference both original and revised versions without collisions! Section \ref{original-section}: \nameref{original-section} of the original is NOT Section \ref{revised-section}: \nameref{revised-section} of the revised manuscript!

\end{enumerate}

\end{letter}

\bibliographystyle{unsrt}
\bibliography{bibliography}

\end{document}